\subsection{Root-Finding}

The function that has been used for the root-finding algorithm is $N(x) = \frac{y}{2}$, where $N(x)$ is the function mentioned in the previous section and $y$ is the maximum of $N(x)$. This maximum can be found in a similar analytical manner as done in section 2.4. Notice that $N(x)$ is very similar to $p(x)$, only multiplied by a factor of $\langle N_{sat} \rangle$. This extra factor, however, can be put out of the braces, and thus the exact same $x$ value for maximum can be obtained from this function, see eq. 10. The equation that has to be solved is thus $N(x) - \frac{b}{2} \left(\frac{a-1}{c}\right)^{1/c} = 0$. It is possible for this equation to have multiple solutions. Therefore, it is necessary to find the number of roots in the function first. This can be done using the given data points, $10^{-4}$, $5$ and the x-value of the maximum of $N(x)$. This last x-value is added to the given data points array and it is then sorted to its right position in the array. The reason to add this point is because the maximum is always above zero, the first and the last points, however, may not be.\footnote{It could also be that the first and last points are positive while still having some roots in between these three values. However, by simply making a rough sketch of the function it can be seen that this is not the case.} By iterating over these points and looking at the sign of the output for the corresponding y-values, it is possible to determine whether the function has passed a root or not.%\footnote{Note that this method only works for roots that cross the y-axis at $y=0$. For example a parabolic function (e.g. $y=x^2$) can have root while not crossing the y-axis at $y=0$. However, from the given function it can be seen that $N(x)$ is always bigger than zero (in the given ranges), and when subtracting the maximum value divided by 2 from this, the function drops which creates the roots that cross the y-axis.}
When having found the number of roots of the function and saving their surrounding x-values, the root coordinates were found with the use of the simple but effective bisection algorithm.\\
The code that has been used is given below:
\lstinputlisting{Q2f.py}
\lstinputlisting[firstline=446,lastline=541]{functions.py}
The code outputs the $x$ and $N(x)$ values of the roots of the function. In addition, the code outputs a 'steepness test' of the function as well. This is because the bisection algorithm might not work as accurate for functions that go very slow to zero (non-steep functions). So this 'steepness' test is actually to test how accurate the $x$ coordinates of the function might be. It is quite a simple test where simply an extra term is added to the x-coordinate of the root, and check how big $N(x)$ is. 
\lstinputlisting{textfiles/roots.txt}
It can be seen from the steepness test that the function evaluated at the found $x$ coordinates of the roots is relatively close to zero. Evaluating $x + 0.1$ at the roots give for both roots quite high function values. Even when adding/subtracting 1e-12, the function is not nearly as close to zero as the found $x$ roots. Therefore, it is save to say that the bisection algorithm has been successful for root finding  for this given function.