\subsection{3D satellite positions}

It is possible to produce radii that statistically follow the satellite profile $n(x)$. These radii then have the probability distribution that goes as $p(x)\mathrm{d}x = \frac{n(x)4\pi x^2 \mathrm{d}x}{\langle N_{sat}\rangle} = 4\pi A \left(\frac{1}{b}\right)^{a-3} x^{a-1} \exp\left[-\left(\frac{x}{b}\right)^c\right]\mathrm{d}x$. One method to generate random radii that follow this distribution is rejection sampling. For rejection sampling a uniform distribution is layed over the probability distribution $p(x)$ in such a way that, for every $x$, the uniform distribution is higher than $p(x)$. The maximum of the probability density function can be taken as the maximum of the uniform distribution, to make sure that the uniform distribution is always greater or equal to the probability density function. The maximum of this probability density function can be easily found by analytically differentiate the PDF to $x$ and equate it to zero,

\begin{gather*}
\frac{\partial p(x)}{\partial x} = 4\pi A\left(\frac{1}{b}\right)^{a-3} \left\lbrace(a-1)x^{a-1}x^{-1}\exp\left[-\left(\frac{x}{b}\right)^c\right] - x^{a-1}\exp\left[-\left(\frac{x}{b}\right)^c\right]\left(\left(\frac{x}{b}\right)^c cx^{-1}\right)\right\rbrace\\
=4\pi A\left(\frac{1}{b}\right)^{a-3}x^{a-1}\exp\left[-\left(\frac{x}{b}\right)^c\right]\left\lbrace\frac{a-1}{x} - c\left(\frac{x}{b}\right)^c \frac{1}{x}\right\rbrace\\
= \frac{p(x)}{x}\left\lbrace a-1-x\left(\frac{x}{b}\right)^c\right\rbrace = 0
\end{gather*}
The first term $\frac{p(x)}{x}$ can not be zero since $p(x)$ is only zero for $x \rightarrow 0$ or $x \rightarrow \infty$, however $\frac{0}{0}$ and $\frac{\infty}{\infty}$ are not defined. The second term $a-1-c\left(\frac{x}{b}\right)^c$ can be set to zero. Solving this equation yields
\begin{equation}
x = b\left(\frac{a-1}{c}\right)^{1/c}
\end{equation}

As can be seen from the equation, the analytical maximum depends only on $a$, $b$, and $c$. Now the distribution can be sampled using rejection sampling. The first step is to generate a random number between 0 and 5 for the x value and to subsequently generate a random number between 0 and the maximum of the PDF for the y value. The PDF is then evaluated at the random generated x value which is compared to the generated random y value. If $y<p(x)$, the random generated $x$ is accepted. \\
The code for this algorithm is shown below.
\lstinputlisting{Q2d.py}
\lstinputlisting[firstline=409,lastline=444]{functions.py}
The generated output is a list of a 100 satellite positions  in spherical coordinates $r$ (or $x$), $\phi$, and $\theta$, where $r$ follows the probability distribution described above and $\phi$ and $\theta$ are generated from a uniform distribution and put in the right intervals, $0<\phi<2\pi$ and $0<\theta<\pi$.\footnote{$\phi$ can be generated by using simply $\phi = 2\pi\cdot U(0,1)$, where $U(0,1)$ is a uniform distribution between 0 and 1. However, $\theta$ is generated via $\theta= \cos^{-1}(1-2U(0,1))$.}
\lstinputlisting{textfiles/satel_coords.txt}